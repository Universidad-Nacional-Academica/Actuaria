\usepackage{ragged2e} % Para justificar
\usepackage{hyperref}
\usepackage[spanish]{babel}
\usepackage[utf8]{inputenc}
\usepackage{bm} 
\usepackage{amssymb}
\usepackage{cancel}  
\usepackage{multicol}
\usepackage{multirow}
\usepackage{undertilde}    
\usepackage{verbatim}
\usepackage{enumitem}
\usepackage{dsfont}
\usepackage{xcolor}
\usepackage{float}
\usepackage{graphicx}
\usepackage{ textcomp }
\usepackage{mathtools} %preíndices \prescript{14}{2}{\mathbf{C}}
\decimalpoint
\selectlanguage{spanish}
\newcommand{\benum}{\begin{enumerate}[label=(\alph*)]}
\newcommand{\eenum}{\end{enumerate}}
\pagenumbering{arabic}
\usepackage{tikz}
\usetikzlibrary{positioning,shapes,shadows,arrows}
\tikzstyle{abstract}=[rectangle, draw=black, rounded corners, fill=blue!40, drop shadow,
        text centered, anchor=north, text=white, text width=3cm]
\tikzstyle{comment}=[rectangle, draw=black, rounded corners, fill=green, drop shadow,
        text centered, anchor=north, text=white, text width=3cm]
\tikzstyle{myarrow}=[->, >=open triangle 90, thick]
\tikzstyle{line}=[-, thick]

